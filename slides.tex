\documentclass{beamer}
\usepackage{lmodern}
\usepackage[frenchb]{babel}
\usepackage[T1]{fontenc}
\usepackage[utf8]{inputenc}
\usepackage{graphicx}

\addtobeamertemplate{navigation symbols}{}{%
\usebeamerfont{footline}%
\usebeamercolor[fg]{footline}%
\hspace{1em}%
\insertframenumber/\inserttotalframenumber
}


\usetheme{Warsaw}

\title{A Necessary and Sufficient Condition for
Consensus Over Random Networks}
\author{Jeremy Krebs - Guillaume Soulié}
\institute{Université Paris Saclay}
\date{\today}

\begin{document}

\begin{frame}
\titlepage
\end{frame}

% --------- Sommaire ---------
\begin{frame}
  \tableofcontents
\end{frame}      
% ----------------------------



\section{Introduction}
		\subsection{Généralités}
		
\begin{frame}
	\begin{itemize}
		\item Problème du Consensus 
		\pause
		\item Temps discret 
		\pause
		\item Systèmes stochastiques
	\end{itemize}
	
	\pause
	
	Problème de la forme:
	$ x_{k+1} = W_k . x_k $

	\pause
	
	$W_k$ peut être vu comme la matrice de poids d'arêtes d'un graphe aléatoire.
\end{frame}

	\subsection{État de l'art}

\begin{frame}
		Le problème de consensus suscite beaucoup d'intérêt:
		 \begin{itemize}
		 	\item Coordination d'agents autonomes
		 	\pause
		 	\item Calculs de moyennes d'un groupe de capteurs
		 	\pause
		 	\item Problèmes de rendez-vous
		 \end{itemize}
		 
		 \pause
		 
		 Cependant ces problèmes considéraient un système déterministe.
\end{frame}

\begin{frame}
	Il y a déjà eu plusieurs résultats sur le sujet:
	\begin{itemize}
		\item $x_k$ converge presque sûrement si les arêtes de $G(W_k)$ sont choisies de manière indépendante et avec la même probabilité,
		\pause
		\item La convergence en probabilité a été prouvée dans le cas d'un modèle avec des arêtes orientées et non-nécessairement indépendantes, avec une hypothèse un peu forte sur les matrices.
	\end{itemize}
\end{frame}

	\subsection{Objectif}

\begin{frame}
	Objectif:
	\begin{itemize}
		\item Trouver une condition nécessaire et suffisante pour un consensus presque sûr
		\item Cas particulier d'un système dynamique qui converge vers un vecteur fixe avec probabilité 1
	\end{itemize}
\end{frame}
				
\begin{frame}
	On considère que:
	\begin{itemize}
		\item Les $W_k$ sont des matrices stochastiques indépendantes et identiquement distribuées (i.i.d.)
		\item
	\end{itemize}	
\end{frame}
		
\section{Définitions}
\begin{frame}
	$ x_k = W_k(\omega)x_{k-1}$ où $W_k$ est une matrice de poids.
	
	\begin{itemize}
		\item $j$ a accès à $i$ si $(i,j)$ est une arête,
		\item $i$ et $j$ communiquent si $(i,j)$ et $(j,i)$ sont des arêtes,
		\item La relation de communication est une relation d'équivalence qui permet de définir de regrouper les arêtes par classes d'équivalence.
	\end{itemize}
\end{frame}

\section{Ergodicité}
\begin{frame}
	$x_k = W_k ... W_1\ x_0$
	
	\bigbreak
	
	Intérêt d'étudier le produit infini de matrices $W_i$ et son ergodicité.
	
	\bigbreak
		
	Notons $U^{(k, p)} = W_{p+k}...W_{p+1}$ le produit à gauche des matrices de la séquence.
\end{frame}

\begin{frame}
	\textbf{Définition - Ergodicité faible:}
	
	La séquence infinie $W_1, W_2, ...$ est faiblement ergodique si et seulement si
	
	$ \forall\ i, j, s = 1,..,n\ $ et $\forall\ p > 0\ (U_{i,s}^{k,p} - U_{j,s}^{k,p}) \rightarrow 0$ quand $k \rightarrow \infty$
	
	\bigbreak

	\pause
	
	\textbf{Définition - Ergodicité forte:}
	
	La séquence infinie $W_1, W_2, ...$ est faiblement ergodique si et seulement si
	
	$ \forall\ i, s = 1,..,n\ U_{i,s}^{k,p} \rightarrow d_s^p$ quand $k \rightarrow \infty$
	
	où $d_s^p$ est une constante ne dépendant pas de $i$
	
\end{frame}

\section{Résultats}

\end{document}
